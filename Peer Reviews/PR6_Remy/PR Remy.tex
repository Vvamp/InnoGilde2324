%!TEX TS-program = xelatex
%!TEX encoding = UTF-8 Unicode
\documentclass[a4paper]{article}
%\usepackage[date=short,backend=biber]{apa}
\usepackage[hidelinks]{hyperref}
\usepackage{cite}
\usepackage[dutch]{babel}
\usepackage[a4paper, left=1in, right=1in, top=1in, bottom=.8in]{geometry}
\usepackage[utf8]{inputenc}
\usepackage{fancyhdr}
\usepackage{titlesec}
\usepackage{geometry}
\usepackage{graphicx}
\usepackage{etoolbox}
\usepackage{listings}
\usepackage{xcolor}
\usepackage{nameref}
\usepackage{tcolorbox}
\usepackage{textcomp}
\usepackage{helvet}
\usepackage{enumitem}
\usepackage{tabularx}
\usepackage{pgf-pie}  
\usepackage{float}
\usepackage{pgfplots}
% Styling
\pagestyle{fancy}
\patchcmd{\chapter}{\thispagestyle{plain}}{\thispagestyle{fancy}}{}{}

\fancyhf{}
\fancyhead[L]{Reviewer: Vincent van Setten }
\fancyhead[R]{Beoordeelde: Remy de Bruijn}
\fancyhead[C]{Peer Review}
\fancyfoot[R]{\thepage}

\titleformat{\chapter}[hang]
{\normalfont\huge\bfseries}{\thechapter.}{10pt}{\huge}
\titlespacing{\chapter}{0pt}{-30pt}{20pt}

\setlength{\parindent}{0.2em}

\textwidth=400pt
\geometry{
    left=25mm
}

\renewcommand{\contentsname}{Inhoudsopgave}



\definecolor{codegreen}{rgb}{0,0.6,0}
\definecolor{codegray}{rgb}{0.5,0.5,0.5}
\definecolor{codepurple}{rgb}{0.58,0,0.82}
\definecolor{backcolour}{rgb}{0.95,0.95,0.92}

\lstdefinestyle{mystyle}{
    backgroundcolor=\color{backcolour},   
    commentstyle=\color{codegreen},
    keywordstyle=\color{magenta},
    numberstyle=\tiny\color{codegray},
    stringstyle=\color{codepurple},
    basicstyle=\ttfamily\footnotesize,
    breakatwhitespace=false,         
    breaklines=true,                 
    captionpos=b,                    
    keepspaces=true,                 
    numbers=left,                    
    numbersep=5pt,                  
    showspaces=false,                
    showstringspaces=false,
    showtabs=false,                  
    tabsize=2
}

\lstset{style=mystyle}

% Commands
\newcommand{\teambox}{
  \begin{tcolorbox}[hbox, colback=blue!5!white,colframe=blue!75!black,
    left=.1mm, right=.1mm, top=.1mm, bottom=.1mm, fontupper=\scriptsize\sffamily]
    Team Keuze
  \end{tcolorbox}
}

\newcommand{\personalbox}{
  \begin{tcolorbox}[hbox, colback=green!5!white,colframe=green!75!black,
    left=.1mm, right=.1mm, top=.1mm, bottom=.1mm, fontupper=\scriptsize\sffamily]
    Persoonlijke Keuze
  \end{tcolorbox}
}
\newcommand{\teamchoice}[1]{
  \section[ #1 ]{#1~\mbox{\raisebox{-2.5pt}{\teambox}}}
}

\newcommand{\personalchoice}[1]{
  \section[ #1 ]{#1~\mbox{\raisebox{-2.5pt}{\personalbox}}}
}

\newcommand{\timestamp}[1]{
  \mbox{\scriptsize \textbf{Datum:} #1} \smallbreak
}

% Document
\begin{document}


% Title Page
\begin{titlepage}
  \begin{center}
      \vspace*{.9cm}
      \Huge
      \textbf{ Peer Review }\\
      \vspace{0.2cm}
      \small \textbf{Datum:} \today \\
      \small TI Gilde, Groep D \\

      \vspace{2cm}
      \normalsize
      \vspace{1cm}
      \Large
      \textbf{In opdracht van}\\
      \large
      \textbf{Hogeschool Utrecht} \\
      \includegraphics[width=0.2\textwidth]{Images/logouni.png}
      \vfill

      \begin{minipage}{0.45\textwidth}
        \large
        \textbf{Reviewer}\\
        \normalsize
        \textbf{Student:} Vincent van Setten \\
        \textbf{Gilde:} TI Gilde, Groep D\\
        \textbf{Innovation Team:} FairPhone (499) \\
        \vspace{2cm}
      \end{minipage}
      \hfill
      \begin{minipage}{0.45\textwidth}
        \large
        \textbf{Beoordeelde}\\
        \normalsize
        \textbf{Student} Remy de Bruijn  \\
        \textbf{Gilde:} TI Gilde, Groep D\\
        \textbf{Innovation Team:} FairPhone (499) \\
        \textbf{Document Versie:} 3.0 \\
        \vspace{2cm}
      \end{minipage}
    \end{center}
\end{titlepage}


\tableofcontents
\clearpage

\section{Inleiding}
Dit document bevat mijn peer review van het verantwoordingsdocument geschreven door Remy de Bruijn, die betrokken is bij het project "FairPhone". 
Het primaire doel van deze review is om een constructieve analyse te bieden die verder gaat dan alleen een algemene indruk. 
Ik zal kijken naar zowel de technische als conceptuele aspecten van het document. Ook zal ik suggesties en potentiële feedback bieden op het project als geheel, wanneer dit naar mijn mening noodzakelijk is.
\par \smallskip
In lijn met de beoordelingscriteria zal mijn focus niet alleen liggen op de grammatica en spelling van het document, maar zal ik ook kritisch kijken naar de helderheid, consistentie en volledigheid van de inhoud.
Hiermee hoop ik de kwaliteit van het document, en mogelijk ook het project, te verbeteren.

\section{Gedetailleerde Review}
\subsection{Structuur}
Je document is mooi gestructureerd. Je versiebeheer is kort en duidelijk. 
Wel zou je kunnen overwegen om ergens te noteren welke feedback je exact hebt verwerkt, zodat ook daar wat gerichter feedback op gegeven kan worden.
Verder is de structuur van je document duidelijk en logisch. Super mooi.

\subsection{Inhoud}
Inhoudelijk ziet je document er ook goed uit, er is een mooie basis voor de verantwoording van je keuzes.
Je zou nog de volgende punten kunnen meepakken.
\begin{enumerate}
  \item Bepaalde alternatieven beter onderbouwen, zoals waarom de besturingssysteem mogelijkheden specifiek die vier waren (en geen anderen). Dit is om beter aan de rubriek te voldoen. Hierin staat namelijk dat je beoordeeld wordt op criteria voor je keuzes en uitleg voor alternatieven en waarom hier wel/niet voor gekozen is.
  \item Je criteria duidelijker maken voor je keuzes, zodat het ook duidelijker wordt wat voor jou van belang is geweest(en waarom) en hoe dat jouw keuze heeft beïnvloed.
  \item Je zou bij hoofdstuk 3.0 bij het kopje 'programmeertaal' kunnen toelichten waarom geprogrammeerd gaat worden in C/C++ en bash.
\end{enumerate}

\subsection{Conclusie}
Je verantwoordingsdocument is duidelijk en logisch. 
je gemaakte keuzes zijn voor mij gelijk duidelijk. Wel zou je nog je alternatieven en criteria beter kunnen toelichten.


% \section{Bibliografie}
% % \nocite{*} % This includes all entries from the .bib file, even if they're not cited in the document
% \begingroup
% \renewcommand{\chapter}[2]{} % Removes the 'Chapter' heading
% \renewcommand{\addcontentsline}[3]{} % Prevents adding this specific entry to TOC
% \bibliographystyle{ieeetr}
% \bibliography{bronnen}
% \endgroup

% \chapter{Bijlagen}
% \textit{Ruimte voor peer reviews}

\end{document}